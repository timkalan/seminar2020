\documentclass[a4paper, 12pt]{article}
\usepackage[slovene]{babel}
\usepackage[utf8]{inputenc}
\usepackage[T1]{fontenc}
\usepackage{lmodern}
\usepackage{blindtext}

\author{Tim Kalan \\ Fakulteta za matematiko in fiziko}
\title{
    Terminske pogodbe o obrestni meri \\ 
    \large (angl. \textit{Interest rate futures})}
\date{19.\ marec\ 2020} % \today za današnji datum 

\begin{document}
\maketitle

\pagebreak

\begin{abstract}
    Še pride.
\end{abstract}

\pagebreak

\tableofcontents

\pagebreak

\section{Uvod}
Ob obsegu današnjega trga in vseh možnostih, ki jih ponuja, se je morda smiselno vprašati,
zakaj bi investitor sploh trgoval z finančnimi instrumenti o obrestni meri. Pojavijo se trije
glavni razlogi: Prvič, tipično je trgovanje na trgu obrestnih mer cenejše in posledično nas 
stane manj, da preko obrestnih mer omilimo tveganje. Prav tako lahko tipično hitreje prilagajamo svoj 
portfelj. Zadnji pomemben razlog pa je, da se obresne mere manj odzovejo na povpraševanje v 
smislu dviganja njihovih cen (oz. cen poslovanja z njimi), kot alternative na trgu denarja.

\section{Trgovanje s terminskimi pogodbami}
V tem razdelku si bomo natančno pogledali principe trgovanja s terminskimi pogodbami. Ugotovili
bomo, kako se zapre pozicijo, kaj je vloga klirinške hiše pri trgovanju in zakaj trgovci potrebujejo
kritje.

S terminski pogodbami se trguje na organiziranem trgu (torej borzah) in zato so standardizirane. 
To pomeni, da se v njih natančno določi vrsta in kvaliteta blaga.

\subsection{Terminske pogodbe}
Na kratko ponovimo nekaj splošnih reči o terminski pogodbi (angl. \textit{\textbf{future}}), da se 
bomo v nadaljevanju lažje sporazumevali.

Terminska pogodba je natančen dogovor med prodajalcem in kupcem, v kateri se določi vrsta
blaga, datum izročitve tega blaga (angl. \textit{delivery date}) in seveda cena, 
ki jo bo plačal kupec - izročitvena cena (angl. \textit{futures price}), ob izteku pogodbe. 
Ob sklenitvi pogodbe ($t = 0$) torej ni nobenih denarnih tokov.
Osnovno premoženje (angl. \textit{the underlying}) v tej pogodbi je lahko praktično karkoli: 
razno blago, devizni tečaji, delnice, indeksi ali pa recimo naš fokus - obrestne mere. 
Terminske pogodbe na obrestne mere spadajo pod \textbf{finančne} terminske pogodbe 
(angl. \textit{financial futures}).

Kupcu terminske pogodbe pogosto rečemo dolga stran in realizira profit, če se cena na trgu 
povzdigne. Na drugi strani pa imamo prodajalca ali kratko stran, ki profitira ob padcu cene.

\subsection{Zapiranje pozicije}
Velika večina finančnih terminskih pogodb ima izročitvene datume marca, junija, septembra ali 
decembra, zato je relativno enostavno določiti, kdaj je potrebno prenehati s trgovanjem s temi 
pogodbami na sekundarnem trgu. Do tega datuma pa ima trgovalec (angl. \textit{trader}) nasljednjo
možno strategijo za zaprtje svoje pozicije: če je kupec, mora prodati isto število identičnih
terminskih pogodb in obratno, če je prodajalec.

Pozicijo lahko zapre tudi na datum izročitve, in sicer tako, da pač izvede svoje obveznosti. 
Kupec torej sprejme osnovno premoženje in ga plača po zmenjeni ceni in obratno za prodajalca - 
ta preda osnovno premoženje in dobi denar. 

Nekatere terminske pogodbe o obrestni meri je mogoče poravnati le z denarjem ali z denarjem 
enakovredni stvari. Take pogodbe v angleščini imenujemo \textit{cash settlement contracts}.

\subsection{Vloga klirinške hiše}
Vsaka borza, kjer se trguje s terminskimi pogodbami in vsaka borza nasploh ima svojo \textbf{
klirinško hišo}. V tem razdelju si bomo ogledali njene funkcije.

Njena glavna funkcija je, da zagotovi, da bosta obe strani pogodbe izpolnjeni in s tem 
udeleženim v pogodbi omili skrbi in zmanjša njihovo tveganje. To doseže tako, da kadarkoli
se pojavi kupec ali prodajalec pogodbe, ona zavzame njegovo nasprotno pozicijo v pogodbi.
Torej, po prvotnem dogovoru med obema stranema se ona ">vrine"< kot kupec za prodajalca in 
prodajalec za kupca. To drastično omili skrbi glede nemoči izvedbe obveznosti nasprotne
strani. 

Prav tako je zaradi klirinške hiše investitorjem bolj enostavno zapreti svoje pozicije, saj 
jim ni treba vključevati nasprotne strani, saj vsi posli potekajo preko hiše. 

\subsection{Zahteve po kritju}
Da investitor sploh lahkto trguje s terminskimi pogodbami, mora odpreti vzdrževalni račun. 
Ko sklene pogodbo, je v njej poleg zgoraj naštetih stvari opredeljena še višina začetnega 
kritja (angl. \textit{initial margin}). To je pač neka količina denarja, ki jo je treba 
naložiti kot depozit za pogodbo (spet v vlogi zmanjševanja tveganja). To kritje ni nujno
v denarju, plačamo lahko namreč tudi s karerimkoli vrednostnim papirjem, ki izplačuje obresti
(npr. obveznica).

V času med sklenitvijo in iztekom pogodbe, njena cena na trgu niha. Ker trgujemo na organiziranem
trgu, je ob koncu vsakega trgovalnega dne pogodba vrednostena (angl. \textit{marking-to-market}). 
Kot kupec pogodbe smo dolžni vzdrževati razliko med stanjem na računu in vzdrževalnim kritjem
(angl. \textit{maintenance margin}) v primeru neugodnega premika cene in imamo pravico vzeti zase 
dodatno maržo, če je premik ugoden. 

Dodatnem kritju, ki ga moramo nakazati pravimo variacijsko kritje (angl. \textit{variation margin}) 
in mora biti nujno v denarju (in ne vrednostnem papirju).

\section{Terminski posli}
Terminski poseli (angl. \textit{forwards}) so zelo podobni terminskim pogodbam s ključno razliko:
niso standardizirani (podorobnosti vsakega posla so dogovorjene na individualnem nivoju) in z njimi
se trguje na neorganiziranem trgu (angl. \textit{over-the-counter}). Posledično je sekundarni trg za 
posle praktično neobstoječ. Obračun cene se lahko zgodi vsak dan, a to ni nujno, če se zmenimo drugače.
Prav tako so generalno bolj tvegani, saj ni klirinške hiše, ki bi zagotovila, da obe strani izpolnita
dogovor. Tveganju, ki nastane, pravimi \textit{bilateral counterparty risk.}

\section{Terminske pogodbe o obrestni meri}
So terminske pogodbe katerih osnovno premoženje izplačuje obresti. 

Terminske pogodbe o obrestni meri lahko glede na datum dospetja njihovega osnovnega premoženja delimo na 
dve skupini: Če je ta datum manj kot leto v prihodnosti, govorimo o kratkotrajnih pogodbah (angl. 
\textit{short-term}),  v nasprotnem primeru pa dolgotrajnih pogodbah (angl. \textit{long-term}). 



V nadaljevanju se bomo osredotočili predvsem na pogodbe na državne obveznice in zakladne menice; s temi
se namreč najbolj pogosto trguje v ZDA, ostale države sveta pa so opisane koncepte posvojile in po njih
modelirale svoje trge s tovrstnimi finančnimi instrumenti.

\subsection{Terminske pogodbe na državne obveznice}
Z njimi se trguje na borzi v Chicagu (\textit{CBOT}). Osnovno premoženje v taki pogodbi je \textdollar
$100.000$

\subsubsection{Conversion factors}
\subsubsection{Cheapest-to-deliver issue}
\subsubsection{Other delivery options}
\subsubsection{Delivery procedure}

\subsection{Terminske pogodbe na zakladno menico \\ (angl. \textit{treasury note futures})}
\subsection{Agency note futures contract}
\subsection{Terminske pogodbe na EURIBOR in LIBOR}

\section{Viri}
\begin{description}
    \item a dela?
\end{description}

\end{document}