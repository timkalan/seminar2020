\documentclass[a4paper, 12pt]{article}
\usepackage[slovene]{babel}
\usepackage[utf8]{inputenc}
\usepackage[T1]{fontenc}
\usepackage{lmodern}
\usepackage{blindtext}

\author{Tim Kalan \\ Fakulteta za matematiko in fiziko}
\title{
    Terminske pogodbe o obrestni meri \\ 
    \large (angl. \textit{Interest rate futures})}
\date{19.\ marec\ 2020} % \today za današnji datum 

\begin{document}
\maketitle
\pagebreak
\tableofcontents
\pagebreak

\begin{abstract}
    Še pride.
\end{abstract}

\section{Uvod}
Ob obsegu današnjega trga in vseh možnostih, ki jih ponuja, se je morda smiselno vprašati,
zakaj bi investitor sploh trgoval z finančnimi instrumenti o obrestni meri. Pojavijo se trije
glavni razlogi: Prvič, tipično je trgovanje na trgu obrestnih mer cenejše in posledično nas 
stane manj, da preko obrestnih mer omilimo tveganje. Prav tako lahko hitreje prilagajamo svoj 
portfelj. Zadnji pomemben razlog pa je, da se obresne mere manj odzovejo na povpraševanje v 
smislu dviganja njihovih cen (oz. cen poslovanja z njimi), kot alternative na trgu denarja.

\section{Trgovanje s terminskimi pogodbami}
V tem razdelku si bomo natančno pogledali principe trgovanja s terminskimi pogodbami. Ugotovili
bomo, kako se zapre pozicijo, kaj je vloga klirinške hiše pri trgovanju in zakaj trgovci potrebujejo
kritje.

S terminski pogodbami se trguje na organiziranem trgu (torej borzah) in zato so standardizirane. 
To pomeni, da se v njih natančno določi vrsta in kvaliteta blaga.

\subsection{Terminske pogodbe}
Na kratko ponovimo nekaj splošnih reči o terminski pogodbi (angl. \textit{\textbf{future}}), da se 
bomo v nadaljevanju lažje sporazumevali.

Terminska pogodba je natančen dogovor med prodajalcem in kupcem, v kateri se določi vrsta
blaga, datum izročitve tega blaga (angl. \textit{delivery date}) in seveda cena, 
ki jo bo plačal kupec - izročitvena cena (angl. \textit{futures price}). 
Osnovno premoženje (angl. \textit{the underlying}) v tej pogodbi je lahko praktično karkoli: 
razno blago, devizni tečaji, delnice, indeksi ali pa recimo naš fokus - obrestne mere. 
Terminske pogodbe na obrestne mere spadajo pod \textbf{finančne} terminske pogodbe 
(angl. \textit{financial futures}).

\subsubsection{Zapiranje pozicije}
Velika večina finančnih terminskih pogodb ima izročitvene datume marca, junija, septembra ali 
decembra, zato je relativno enostavno določiti, kdaj je potrebno prenehati s trgovanjem s temi 
pogodbami na sekundarnem trgu. Do tega datuma pa ima trgovalec (angl. \textit{trader}) nasljednjo
možno strategijo za zaprtje svoje pozicije: če je kupec, mora prodati isto število identičnih
terminskih pogodb in obratno, če je prodajalec.

Pozicijo lahko zapre tudi na datum izročitve, in sicer tako, da pač izvede svoje obveznosti. 
Kupec torej sprejme osnovno premoženje in ga plača po zmenjeni ceni in obratno za prodajalca - 
ta preda osnovno premoženje in dobi denar. 

\subsubsection{Vloga klirinške hiše}
\subsubsection{Kritje}

\section{Terminski posli}

\section{Tveganje in dobiček}

\section{Terminske pogodbe o obrestni meri}
\subsection{Terminske pogodbe na državne obveznice}
\subsubsection{Conversion factors}
\subsubsection{Cheapest-to-deliver issue}
\subsubsection{Other delivery options}
\subsubsection{Delivery procedure}

\subsection{Terminske pogodbe na zakladno menico \\ (angl. \textit{treasury note futures})}
\subsection{Agency note futures contract}
\subsection{Terminske pogodbe na EURIBOR in LIBOR}

\section{Viri}
\begin{description}
    \item a dela?
\end{description}

\end{document}