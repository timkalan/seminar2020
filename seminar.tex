\documentclass[a4paper, 12pt]{article}
\usepackage[slovene]{babel}
\usepackage[utf8]{inputenc}
\usepackage[T1]{fontenc}
\usepackage{lmodern}
\usepackage{blindtext}

\author{Tim Kalan \\ Fakulteta za matematiko in fiziko}
\title{
    Terminske pogodbe o obrestni meri \\ 
    \large (angl. \textit{Interest rate futures})}
\date{\today} % začel si 19. marec 

\begin{document}
\maketitle
\pagebreak
\tableofcontents
\pagebreak

\begin{abstract}
    Še pride.
\end{abstract}

\section{Uvod}
Ob obsegu današnjega trga in vseh možnostih, ki jih ponuja, se je morda smiselno vprašati,
zakaj bi investitor sploh trgoval z finančnimi instrumenti o obrestni meri. Pojavijo se trije
glavni razlogi: Prvič, tipično je trgovanje na trgu obrestnih mer cenejše in posledično nas 
stane manj, da preko obrestnih mer omilimo tveganje. Prav tako lahko hitreje prilagajamo svoj 
portfelj. Zadnji pomemben razlog pa je, da se obresne mere manj odzovejo na povpraševanje v 
smislu dviganja njihovih cen (oz. cen poslovanja z njimi), kot alternative na trgu denarja.

\section{Terminske pogodbe}
\subsection{Trgovanje s terminskimi pogodbami}
\subsection{Terminski posli}
\subsection{Tveganje}

\section{Exchange-Traded Interest Rate Futures}
\subsection{Terminske pogodbe na državnih obveznicah}
\subsection{Treasury note futures}

\section{Viri}
\begin{description}
    \item a dela?
\end{description}

\end{document}