\documentclass[a4paper]{article}
\usepackage[slovene]{babel}
\usepackage[utf8]{inputenc}
\usepackage[T1]{fontenc}
\usepackage{lmodern}
\usepackage{blindtext}

\author{Tim Kalan \\ Fakulteta za matematiko in fiziko}
\title{
    Terminske pogodbe o obrestni meri \\ 
    \large (angl. \textit{Interest rate futures})}
\date{19.\ marec\ 2020} % \today za današnji datum 

\begin{document}
\maketitle

\pagebreak

\begin{abstract}
    Še pride.
\end{abstract}

\pagebreak

\tableofcontents

\pagebreak

\section{Uvod}
Ob obsegu današnjega trga in vseh možnostih, ki jih ponuja, se je morda smiselno vprašati,
zakaj bi investitor sploh trgoval z finančnimi instrumenti o obrestni meri. Pojavijo se trije
glavni razlogi: Prvič, tipično je trgovanje na trgu obrestnih mer cenejše in posledično nas 
stane manj, da preko obrestnih mer omilimo tveganje. Prav tako lahko tipično hitreje prilagajamo svoj 
portfelj. Zadnji pomemben razlog pa je, da se obresne mere manj odzovejo na povpraševanje v 
smislu dvigovanja njihovih cen (oz. cen poslovanja z njimi), kot alternative na trgu denarja.

\section{Trgovanje s terminskimi pogodbami}
V tem razdelku si bomo natančno pogledali principe trgovanja s terminskimi pogodbami. Ugotovili
bomo, kako se zapre pozicijo, kaj je vloga klirinške hiše pri trgovanju in zakaj trgovci potrebujejo
kritje.

S terminski pogodbami se trguje na organiziranem trgu (torej borzah) in zato so standardizirane. 
To pomeni, da se v njih natančno določi vrsta in kvaliteta blaga.

\subsection{Terminske pogodbe}
Na kratko ponovimo nekaj splošnih reči o terminski pogodbi (angl. \textit{\textbf{future}}), da se 
bomo v nadaljevanju lažje sporazumevali.

Terminska pogodba je natančen dogovor med prodajalcem in kupcem, v kateri se določi vrsta
blaga, datum izročitve tega blaga (angl. \textit{delivery date}) in seveda cena, 
ki jo bo plačal kupec - izročitvena cena (angl. \textit{futures price}), ob izteku pogodbe. 
Ob sklenitvi pogodbe ($t = 0$) torej ni nobenih denarnih tokov.
Osnovno premoženje (angl. \textit{the underlying}) v tej pogodbi je lahko praktično karkoli: 
razno blago, devizni tečaji, delnice, indeksi ali pa recimo naš fokus - obrestne mere. 
Terminske pogodbe na obrestne mere spadajo pod \textbf{finančne} terminske pogodbe 
(angl. \textit{financial futures}), kar preprosto pomeni, da je osnovno premoženje v njih nek
finančni instrument; recimo ti, ki smo jih našteli zgoraj.

Kupcu terminske pogodbe pogosto rečemo dolga stran in realizira profit, če se cena na trgu 
povzdigne. Na drugi strani pa imamo prodajalca ali kratko stran, ki profitira ob padcu cene.

\subsection{Zapiranje pozicije}
Velika večina finančnih terminskih pogodb ima izročitvene datume marca, junija, septembra ali 
decembra, zato je relativno enostavno določiti, kdaj je potrebno prenehati s trgovanjem s temi 
pogodbami na sekundarnem trgu; CBOT namreč določi tudi to. Do tega datuma pa ima trgovalec 
(angl. \textit{trader}) naslednjo možno strategijo za zaprtje svoje pozicije (naj opomnimo, da 
zapiranje pozicije pomeni izstop iz posla. Ko pozicijo zapremo, vemo točno kakšne denarne 
tokove je ustvarila): če je kupec, mora prodati isto število identičnih terminskih pogodb in 
obratno, če je prodajalec.

Pozicijo lahko zapre tudi na datum izročitve, in sicer tako, da pač izvede svoje obveznosti. 
Kupec torej sprejme osnovno premoženje in ga plača po zmenjeni ceni in obratno za prodajalca - 
ta preda osnovno premoženje in dobi denar. 

Nekatere terminske pogodbe o obrestni meri je mogoče poravnati le z denarjem ali z denarjem 
enakovrednimi stvarmi. Take pogodbe v angleščini imenujemo \textit{cash settlement contracts}.

\subsection{Vloga klirinške hiše}
Vsaka borza, kjer se trguje s terminskimi pogodbami in vsaka borza nasploh ima svojo \textbf{
klirinško hišo}. V tem razdelju si bomo ogledali njene funkcije.

Njena glavna funkcija je, da zagotovi, da bosta obe strani pogodbe izpolnjeni in s tem 
udeleženim v pogodbi omili skrbi in zmanjša njihovo tveganje. To doseže tako, da kadarkoli
se pojavi kupec ali prodajalec pogodbe, ona zavzame njegovo nasprotno pozicijo v pogodbi.
Torej, po prvotnem dogovoru med obema stranema se ona ">vrine"< kot kupec za prodajalca in 
prodajalec za kupca. To drastično omili skrbi glede nemoči izvedbe obveznosti (angl. 
\textit{default}) nasprotne strani. 

Prav tako je zaradi klirinške hiše investitorjem bolj enostavno zapreti svoje pozicije, saj 
jim ni treba vključevati nasprotne strani, ker vsi posli potekajo preko hiše. 

\subsection{Zahteve po kritju}
Da investitor sploh lahko trguje s terminskimi pogodbami, mora odpreti vzdrževalni račun. 
Ko sklene pogodbo, je v njej poleg specifikacij glede osnovnega premoženja, datuma in cene 
opredeljena še višina začetnega kritja (angl. \textit{initial margin}). To je pač neka 
količina denarja, ki jo je treba naložiti kot depozit za pogodbo (spet v vlogi zmanjševanja 
tveganja). To kritje ni nujno v denarju (oz. denarnih ekvivalentih), plačamo lahko namreč 
tudi s katerimkoli vrednostnim papirjem, ki izplačuje obresti (npr. obveznica).

V času med sklenitvijo in iztekom pogodbe, njena cena na trgu niha. Ker trgujemo na organiziranem
trgu, je ob koncu vsakega trgovalnega dne pogodba vrednostena (angl. \textit{marking-to-market}). 
Kot kupec pogodbe smo dolžni vzdrževati razliko med stanjem na računu in vzdrževalnim kritjem
(angl. \textit{maintenance margin}) v primeru neugodnega premika cene in imamo pravico vzeti zase 
dodatno oz. presežno maržo, če je premik ugoden. 

Dodatnemu kritju, ki ga moramo nakazati pravimo variacijsko kritje (angl. \textit{variation margin}) 
in mora biti nujno v denarju (in ne vrednostnem papirju). 

\section{Terminski posli}
Terminski posli (angl. \textit{forwards}) so zelo podobni terminskim pogodbam s ključno razliko:
niso standardizirani, kar pomeni, da so podrobnosti vsakega posla dogovorjene na individualnem nivoju 
in z njimibse trguje na neorganiziranem trgu (angl. \textit{over-the-counter}). Posledično je sekundarni 
trg za posle praktično neobstoječ. Obračun cene se lahko zgodi vsak dan, a to ni nujno, če se zmenimo 
drugače. Prav tako so generalno bolj tvegani, saj ni klirinške hiše, ki bi zagotovila, da obe strani 
izpolnita dogovor. Tveganju, ki nastane, pravimo \textit{bilateral counterparty risk}. V ostalih pogledih
in logiko, ki je ozadju, pa so praktično enaki terminskim pogodbam, zato koncepti, ki so opisani v 
nadaljevanju seveda veljajo tudi za njih.

\section{Terminske pogodbe o obrestni meri}
So terminske pogodbe, katerih osnovno premoženje izplačuje obresti. Kupcu pogodbe tako omogočajo, da 
ustvari dobiček (ali izgubo) zaradi sprememb obrestnih mer. S pametnim razmislekom in uporabo prave 
strategije se lahko z njimi zavaruje pred spremembo obrestnih mer, z dodatkom sreče pri napovedovanju
trga pa lahko tudi dobro zasluži.

% če ne omeniš kasneje, daš lahko sem simpl primer o posojilu s spremenljivo mero in kako se s tem zavarujemo
% pred izgubo, če mera naraste.

Terminske pogodbe o obrestni meri lahko glede na datum dospetja njihovega osnovnega premoženja delimo na 
dve skupini: Če je ta datum manj kot leto v prihodnosti, govorimo o kratkotrajnih pogodbah (angl. 
\textit{short-term}), v nasprotnem primeru pa o dolgotrajnih pogodbah (angl. \textit{long-term}). 

V nadaljevanju se bomo osredotočili predvsem na pogodbe na državne obveznice in zakladne menice; s temi
se namreč najbolj pogosto trguje v ZDA, ostale države sveta pa so opisane koncepte posvojile in po njih
modelirale svoje trge s tovrstnimi finančnimi instrumenti. Mi bomo natančno pogledali, kako se trguje na
borzi CBOT, a večina mehanizmov v ozadju je podobna tudi na drugih borzah, kjer se z njimi trguje.

\subsection{Terminske pogodbe na državne obveznice}
Z njimi se trguje na borzi v Chicagu (\textit{CBOT oz. Chicago board of trade}); zato so tudi veliko 
bolj popularne v ZDA, kot pri nas. Osnovno premoženje v taki pogodbi je hipotetična dvajset-letna 
kuponska obveznica z vrednostjo \textdollar$100.000$. Kuponski obrestni meri v tem primeru pravimo 
obračunski kupon (angl. \textit{notional coupon}). Pogodbe so kotirane tako, da nominalno vrednost 
obveznice označimo s $100$ (kar pomeni $100\%$ vrednosti). Razlike so potem kotirane v dvaintridesetinah 
enega procenta. To se tradicionalno označi s 
%
$$ celi\:del - število\:dvaintridesetin, $$
%
to pomeni, da $97-16$ predstavlja $97,5\%$ nominalne vrednosti. To je potem tudi cena, ki je zapisana v 
pogodbi.

Na borzi je definirano tudi minimalno odstopanje cene (angl. \textit{minimum price fluctuation}) obveznice
kot $1/32$ enega procenta, kar se pri dani nominalni vrednosti prevede v \textdollar$31,25.$ To pomeni, da
se cena zaokrožuje s tako natančnostjo.

\subsubsection{Hipotetične obveznice}
Zgoraj smo omenili, da je osnovno premoženje hipotetična obveznica. To je zato, ker ob sklenitvi pogodbe 
sploh ni potrebno, da kratka stran ima pripravljeno dejansko obveznico za prodajo ob datumu izročitve.
To je dovoljeno zato, ker vedno obstaja možnost predčasnega zapiranja pozicije. 

Če pa se dejansko odloči za izvedbo, pa mora predati obveznico. Le-to izbere iz množice obveznic, ki jih
CBOT določi za primerne. Eden izmed pogojev je, da imajo obveznice do datuma zapadlosti na datum izročitve
vsaj 15 let, drugače pa se zahteva, da so vsaj podobne navedeni hipotetični obveznici, ker ni nujno, da na
trgu obstaja identična obveznica. 

Ta aspekt je tudi ena izmed najbolj zanimivih stvari pri tej vrsti pogodbe, zato si bomo v nadaljevanju 
pogledali nekaj detajlov.

% mogoče en primer tukaj; tista tabela iz vira recimo

\subsubsection{Pretvorbeni faktorji}
Zaradi nenujosti obstoja primerne obveznice, pred prodajo pogodb z določenim datumom zapadlosti CBOT 
določi pretvorbene faktorje (angl. \textit{conversion factors}), ki naredijo te pogodbe pravične. 
Odvisni so od dejanske obveznice, ki jo dolga stran da, in od izročitvenega datuma. Če torej prodajalec
da dejansko bolj donosno obveznico, kot je v pogodbi, bo CBOT tako zagotovil, da vseeno dobi pravično
ceno (in obratno, če da manj donosno obveznico).

% simpl primer o bolj in manj donosni obveznici

Dejanska cena, ki jo kupec plača (angl. \textit{invoice price}) je tako odvisna od mnogih stvari. 
Seveda je pomembna izročivena cena, določena v sami pogodbi in ">velikost pogodbe"< (torej število
obveznic, ki se prodajo s pogodbo). Zgoraj smo ugotovili, da je pomemben tudi pretvorbeni faktor. 
V osnovi pa govorimo o instrumentih na obrestne mere, zato ne smemo pozabiti na natečene obresti, 
ki jih prinaša naše osnovno premoženje. Če povzamemo:
%
$$ cena = št.\:obveznic \times K \times pretvorbeni\:faktor + natečene\:obresti $$
%

% mogoče še en primer tle

\subsubsection{Obveznica, najcenejša za poravnavo}
Zgoraj smo ugotovili, da je možno pri predaji izbirati med različnimi obveznicami, in seveda niso
vse enakovredne. Želja dolge strani je poiskati najcenejšo (angl. \textit{cheapest-to-deliver
issue}) in zato je potrebna analiza vseh možnosti. To izvedemo z naslednjim razmislekom:

Zamislimo si, da istočasno prodamo terminsko pogodbo in z izposojenim denarjem kupimo eno 
izmed obveznic, ki so navedene kot primerne za to pogodbo. Na datum izročive potem predamo 
obveznico in z dobljenim denarjem poplačamo dolg - to imenujemo \textit{cash and carry trade}. 
S tako situacijo potem poznamo dovolj podatkov, da izračunamo stopnjo povračila za tako 
investicijo, pogosto to imenujemo \textbf{implicirana repo stopnja} (angl. \textit{implied 
repo rate}). To lahko izračunamo za vse primerne obveznice in na koncu izberemo tisto, 
katere stopja je najvišja. To obveznico potem imenujemo obveznica, najcenejša za poravnavo 
(angl. \textit{cheapest-to-deliver issue}).

Za izračun implicirane repo stopnje moramo o obveznici poznati naslednje podake:

\begin{itemize}
    \item Ceno po kateri lahko kupimo obveznico in natečene obresti, ki ji pripradajo,
    \item Pretvorjeno ceno in natečene obresti, ki jih bomo dobili ob izročitvi, 
    \item Kupone, ki jih bomo dobili do izročivenega datuma,
    \item Dobiček od reinvesticije kuponov od njihovega izplačila do datuma izročive.
\end{itemize}

Čeprav slednje točke ne moremo točno poznati, imajo njene podrobnosti majhen učinek na 
implicirano stopnjo povračila in zato jih lahko zanemarimo. S tem dobimo naslednjo formulo:
%
$$ implicirana\:repo\:stopnja = \frac{donos}{strošek\:investicije} \times
\frac{360}{dnevi\:do\:izročitve} $$
%
Kjer je donos razlika med prejetim izkupičkom ob izročitvenem datumu in stroškom investicije. 
Izkupiček je torej izročitvena cena terminske pogodbe (ne pozabimo upoštevati pretvorbenega 
faktorja in natečenih obresti), kuponi, ki jih prejmemo in obresti od reinvesticije kuponov. 
%
$$ izkupiček = pretvorjena\:cena + natečene\:obresti + kuponi + obresti $$
%
Kjer velja, da obresti ne poznamo točno, pač pa to ocenimo glede na tržne razmere. Pripomnimo
še, da kupone reinvestiramo za toliko dni, kot mine med njihovim izplačilom in izročitvenim
datumom in da to v formuli merimo kot delež leta (t.j. $360$ dni). 

Strošek investicije je pa torej cena obveznice, ki jo kupimo in natečene obresti, ki jih plačamo.

Zadnji ulomek pa je enostavno anualizacija te repo stopnje, saj to nekako standardizira izračun
in omogoča lažjo primerjavo. 

% tu ponovno lahko pride primer ane

\subsubsection{Izbire pri izročitvi}
Prodajalec ima ob izročitvi pravico in dolžnost sprejeti nekaj pomembnih odločitev. CBOT mu 
zagotavlja, da se lahko odloča vsaj glede naslednjih stvari:

\begin{itemize}
    \item Odločiti se mora, katero izmed primernih obveznic bo izročil. Tej odločitvi v angleščini 
            rečemo \textit{quality/swap option}
    \item Izbere lahko, kater dan v izročitvenem mesecu bo izkoristil za dejansko izročitev
            obveznice - \textit{timing option}.
    \item Lahko si izbrere, da obveznico preda pred ali po zaprtju borze; tako lahko izbere, če 
            bo znana končna cena te pogodbe. Ima tudi primerno ime: \textit{wild card option}. 
\end{itemize}

Zaradi teh odločitev dolga pozicija nikoli tekom pogodbe ne more vedeti točno katero obveznico bo
dobila, niti ne pozna točnega datuma predaje.

\subsubsection{Proces izročitve}
Ko se pogodba izteče (oz. si kratka stran izbere datum), prodajalca še vedno čaka 3-dnevni proces 
izročitve. 

Prvi dan (angl. \textit{position day}) mora do osme ure zvečer po chicaškem času sporočiti na 
CBOT, da ima namen izročiti. 

Drugi dan (angl. \textit{notice day}) si izbere katero obveznico bo predal. To odločitev mora 
sprejeti do štirinajste ure. CBOT potem pregleda vse potencialne kupce, izbere tistega, ki 
ima odprto dolgo pozicijo najdlje in mu do šestnajste ure sporoči, da se bo pogodba izvršila. 

Tretji dan (angl. \textit{delivery day}) se do desete ure izvede dejanska izročitev. Prodajalec 
mora imeti na računu pripravljeno obveznico, kupec pa denar. 


\subsection{Terminske pogodbe na zakladno menico \\ (angl. \textit{treasury note futures})}
CBOT ponuja pogodbe dolžine $2$, $5$ ali $10$ let, ki so vse izpeljanke standardnih pogodb na 
državne obveznice in se še vedno menjajo primarno pri njih. 
Osnovno premoženje pri tej pogodbi je hipotetična $10$-letna zakladna menica z vrednostjo 
\textdollar$100.000$. Za razliko od obveznic, je tu določena tudi kuponska obrestna mera. 
Ponavadi je ta $6\%$. 

\subsubsection{Razlika med zakladno menico in državno obveznico}



\subsection{Agency note futures contract}
\subsection{Terminske pogodbe na EURIBOR in LIBOR}

\section{Viri}
\begin{description}
    \item a dela?
\end{description}

\end{document}