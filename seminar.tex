\documentclass[a4paper, 11pt]{article}
\usepackage[slovene]{babel}
\usepackage[utf8]{inputenc}
\usepackage[T1]{fontenc}
\usepackage{lmodern}
\usepackage{blindtext}
\usepackage{amsmath} % pravilen izpis v "math mode"
\usepackage{hyperref}
\hypersetup{hidelinks}

\newtheorem{primer}{Primer}

\author{Tim Kalan \\ Fakulteta za matematiko in fiziko}
\title{
    Terminske pogodbe o obrestni meri \\ 
    \large (angl. \textit{Interest rate futures})}
\date{19.\ marec\ 2020} % \today za današnji datum 

\begin{document}
\begin{titlepage}
    \maketitle
    \thispagestyle{empty}
\end{titlepage}

\pagebreak

\begin{abstract}
    Finančni instrumenti o obrestni meri nam omogočajo zaščito pred tveganjem premikov obrestnih mer. Za to 
    lahko uporabimo recimo terminsko pogodbo o obrestni meri, ki ima za osnovno premoženje hipotetičen
    instrument (recimo obveznico). Zaradi tega dejstva je prodajalcu prepuščena odločitev, katero 
    obveznico bo dejansko prodal. Iz tu potem izhaja ekonomka analiza vseh možnosti in izračun donosa
    vsake možne izbire. Ko dobimo obveznico, ki je najcenejša za poravnavo, hkrati spoznamo osnovo za 
    vrednotenje teh pogodb. 

    Tekom seminarske naloge spoznamo tudi mehaniko trgovanja s terminskimi pogodbami na ameriški borzi 
    \textit{Chicago board of trade}, ki pa je temelj in podlaga za ostale borze, ki trgujejo s temi pogodbami. 
    Osredotočimo se predvsem na pogodbe na državne obveznice in spoznamo proces poravnave in možnosti, ki jih 
    imamo na voljo pri zapiranju pozicije.

    \textit{Ključne besede: terminska pogodba, CBOT, državna obveznica, donos, poravnava}
    

\end{abstract}

\pagebreak

\tableofcontents

\pagebreak

\section{Uvod}
Ob obsegu današnjega trga in vseh možnostih, ki jih ponuja, se je morda smiselno vprašati,
zakaj bi investitor sploh trgoval z finančnimi instrumenti o obrestni meri. Pojavijo se trije
glavni razlogi: Prvič, tipično je trgovanje na trgu finančnih instrumentov, ki temeljijo na 
obrestnih merah, cenejše in posledično nas stane manj, da preko teh instrumentov omilimo 
tveganje. Prav tako lahko tipično hitreje prilagajamo svoj portfelj. Zadnji pomemben razlog 
pa je, da se instrumenti o obrestni meri manj odzovejo na povpraševanje v smislu dvigovanja 
njihovih cen (oz. cen poslovanja z njimi), kot alternative na trgu denarja.

\section{Trgovanje s terminskimi pogodbami}
S terminski pogodbami se trguje na organiziranem trgu (torej borzah) in zato so standardizirane. 
To pomeni, da se v njih natančno določi vrsta in kvaliteta blaga.

V tem razdelku si bomo natančno pogledali principe trgovanja s terminskimi pogodbami. Ugotovili
bomo, kako se zapre pozicijo, kaj je vloga klirinške hiše pri trgovanju in zakaj trgovci potrebujejo
kritje.

\subsection{Terminske pogodbe}
Na kratko ponovimo nekaj splošnih reči o terminski pogodbi (angl. \textit{\textbf{future}}), da se 
bomo v nadaljevanju lažje sporazumevali.

Terminska pogodba je natančen dogovor med prodajalcem in kupcem, v kateri se določi vrsta
blaga, datum izročitve tega blaga (angl. \textit{delivery date}) in seveda cena, 
ki jo bo plačal kupec - izročitvena cena (angl. \textit{futures price}), ob izteku pogodbe. 
Ob sklenitvi pogodbe ($t = 0$) torej ni nobenih denarnih tokov.
Osnovno premoženje (angl. \textit{the underlying}) v tej pogodbi je lahko praktično karkoli: 
razno blago, devizni tečaji, delnice, indeksi ali pa recimo naš fokus - obrestne mere. 
Terminske pogodbe na obrestne mere spadajo pod \textbf{finančne} terminske pogodbe 
(angl. \textit{financial futures}), kar preprosto pomeni, da je osnovno premoženje v njih nek
finančni instrument; recimo ti, ki smo jih našteli zgoraj.

Kupcu terminske pogodbe pogosto rečemo dolga stran in realizira profit, če se cena pogodbe na 
trgu povzdigne. Na drugi strani pa imamo prodajalca ali kratko stran, ki profitira ob padcu cene.

\subsection{Zapiranje pozicije}
Velika večina finančnih terminskih pogodb ima izročitvene datume marca, junija, septembra ali 
decembra. Do tega datuma pa ima trgovalec (angl. \textit{trader}) naslednjo možno strategijo za 
zapiranje svoje pozicije (naj opomnimo, da zapiranje pozicije pomeni izstop iz posla. Ko pozicijo 
zapremo, vemo točno kakšne denarne tokove je ustvarila): če je kupec, mora prodati isto število 
identičnih terminskih pogodb in obratno, če je prodajalec.

Pozicijo lahko zapre tudi na datum izročitve, in sicer tako, da pač izvede svoje obveznosti. 
Kupec torej sprejme osnovno premoženje in ga plača po zmenjeni ceni in obratno za prodajalca - 
ta preda osnovno premoženje in dobi denar. 

Nekatere terminske pogodbe o obrestni meri je mogoče poravnati le z denarjem ali z denarjem 
enakovrednimi stvarmi. Take pogodbe v angleščini imenujemo \textit{cash settlement contracts}.

\subsection{Vloga klirinške hiše}
Vsaka borza, kjer se trguje s terminskimi pogodbami in vsaka borza nasploh ima sebi asociirano 
\textbf{klirinško hišo}. V tem razdelku si bomo ogledali njene funkcije.

Njena glavna funkcija je, da zagotovi, da bosta obe strani pogodbe izpolnjeni in s tem 
udeleženim v pogodbi omili skrbi in zmanjša njihovo tveganje. To doseže tako, da kadarkoli
se pojavi kupec ali prodajalec pogodbe, ona zavzame njegovo nasprotno pozicijo v pogodbi.
Torej, po prvotnem dogovoru med obema stranema se ona ">vrine"< kot kupec za prodajalca in 
prodajalec za kupca. To drastično omili skrbi glede nemoči izvedbe obveznosti (angl. 
\textit{default}) nasprotne strani. 

Prav tako je zaradi klirinške hiše investitorjem bolj enostavno zapreti svoje pozicije, saj 
jim ni treba vključevati nasprotne strani, ker vsi posli potekajo preko hiše. 

\subsection{Zahteve po kritju}
Da investitor sploh lahko trguje s terminskimi pogodbami, mora odpreti vzdrževalni račun. 
Ko sklene pogodbo, je v njej poleg specifikacij glede osnovnega premoženja, datuma in cene 
opredeljena še višina začetnega kritja (angl. \textit{initial margin}). To je pač neka 
količina denarja, ki jo je treba naložiti kot depozit za pogodbo (spet v vlogi zmanjševanja 
tveganja). To kritje ni nujno v denarju (oz. denarnih ekvivalentih), plačamo lahko namreč 
tudi s katerimkoli vrednostnim papirjem, ki izplačuje obresti (npr. obveznica).

V času med sklenitvijo in iztekom pogodbe, njena cena na trgu niha. Ker trgujemo na organiziranem
trgu, je ob koncu vsakega trgovalnega dne pogodba vrednotena (angl. \textit{marking-to-market}). 
Kot kupec pogodbe smo dolžni vzdrževati količino denarja na računu in poskrbeti, da je le-tega več, 
kot je predpisano vzdrževalno kritje (angl. \textit{maintenance margin}). v primeru neugodnega 
premika cene moramo torej na račun naložiti dodaten denar, da zadovoljimo to zahtevo, temu denarju 
pravimo variacijsko kritje (angl. \textit{variation margin}). Če pa se cena zniža in imamo na 
računu zato preveč kritja, imamo pravico ta presežek vzeti z računa. 

Dolžnost kupca je, da vse spremembe zahtev po kritju poravna v roku $24$ ur, če mu to ne uspe, se
njegova pozicija avtomatično zapre.

\section{Terminski posli}
Terminski posli (angl. \textit{forwards}) so zelo podobni terminskim pogodbam s ključno razliko:
niso standardizirani, kar pomeni, da so podrobnosti vsakega posla dogovorjene na individualnem nivoju 
in z njimi se trguje na neorganiziranem oz. odprtem trgu (angl. \textit{over-the-counter}). Posledično 
je sekundarni trg za posle lahko neobstoječ (to pa ni nujno; nekateri vrednostni papirji so zelo 
likvidni). Obračun cene se lahko zgodi vsak dan, a se zaradi svobode, ki jo imamo pri trgovanju s 
terminskimi posli, lahko zmenimo drugače. Prav tako so generalno bolj tvegani, saj ni klirinške hiše, 
ki bi zagotovila, da obe strani izpolnita dogovor. Tveganju, ki nastane, pravimo dvostransko tveganje 
nasprotne strani (angl. \textit{bilateral counterparty risk}). V ostalih pogledih in logiki, ki je 
ozadju, pa so praktično enaki terminskim pogodbam, zato koncepti, ki so opisani v nadaljevanju seveda 
veljajo tudi za njih.

\section{Terminske pogodbe o obrestni meri}
So terminske pogodbe, katerih osnovno premoženje izplačuje obresti. Kupcu pogodbe tako omogočajo, da 
ustvari dobiček (ali izgubo) zaradi sprememb obrestnih mer. Imajo dve glavni uporabi: poglavitno se 
uporabljajo za zaščito pred premikom obrestnih mer (angl. \textit{hedging}), z uporabo primerne analize 
in modelov pa omogočajo tudi zaslužek. 

\begin{primer}
    Recimo, da imamo izposojen denar po spremenljivi obrestni meri in nas skrbi, da bo obrestna mera v 
    prihodnosti narastla. Če prodamo terminsko pogodbo na državno obveznico, se s tem zavarujemo pred 
    takim porastom. Poglejmo kaj se zgodi:

    Če obrestna mera res naraste, pade cena obveznice, ki je osnovno premoženje naše pogodbe. Zato pade
    tudi cena pogodbe. Ker pa smo mi pogodbo prodali po višji ceni, smo s tem efektivno zaslužili toliko, 
    kot izgubimo zaradi višje obrestne mere. Če pa smo se pri naši napovedi zmotili in mera pade, bo 
    vrednost naše pogodbe padla, to izgubo pa bodo pokrile nižje obresti. S takim poslom tako poskrbimo, 
    da bo začetna obrestna mera veljala celotno obdobje in efektivno spremenimo variabilno obrestno mero 
    v fiksno. 
\end{primer}

Terminske pogodbe o obrestni meri lahko glede na datum dospetja njihovega osnovnega premoženja delimo na 
dve skupini: Če je ta datum manj kot leto v prihodnosti, govorimo o kratkoročnih pogodbah (angl. 
\textit{short-term}), v nasprotnem primeru pa o dolgoročnih pogodbah (angl. \textit{long-term}). 

V nadaljevanju se bomo osredotočili predvsem na pogodbe na državne obveznice in zakladne menice; s temi
se namreč najbolj pogosto trguje v ZDA, ostale države sveta pa so opisane koncepte posvojile in po njih
modelirale svoje trge s tovrstnimi finančnimi instrumenti. Mi bomo natančno pogledali, kako se trguje na
borzi CBOT, a večina mehanizmov v ozadju je podobna tudi na drugih borzah, kjer se z njimi trguje.

\subsection{Terminske pogodbe na državne obveznice}
Z njimi se trguje na borzi v Chicagu (\textit{CBOT oz. Chicago board of trade}); zato so tudi veliko 
bolj popularne v ZDA, kot pri nas. Osnovno premoženje v taki pogodbi je hipotetična dvajset-letna 
kuponska obveznica z vrednostjo \textdollar$100.000$. Kuponski obrestni meri v tem primeru pravimo 
obračunski kupon (angl. \textit{notional coupon}). Pogodbe so kotirane tako, da nominalno vrednost 
obveznice označimo s $100$ (kar pomeni $100\%$ vrednosti). Razlike so potem kotirane v dvaintridesetinah 
enega procenta. To se tradicionalno označi s 
%
$$ celi\:del - število\:dvaintridesetin. $$
%

\begin{primer}
    Recimo, da je v pogodbi zapisana cena $97-16$. To pomeni, da moramo za vsako obveznico, ki jo kupimo 
    na ta način, plačati $97\% + (16/32)\% = 97,5\%$ nominalne vrednosti. 
\end{primer}

Na borzi je definirano tudi minimalno odstopanje cene (angl. \textit{minimum price fluctuation}) obveznice
kot $1/32$ enega procenta, kar se pri dani nominalni vrednosti prevede v \textdollar$31,25.$ To pomeni, da
se cena zaokrožuje s tako natančnostjo.

\subsubsection{Hipotetične obveznice}
Zgoraj smo omenili, da je osnovno premoženje hipotetična obveznica. To je zato, ker ob sklenitvi pogodbe 
sploh ni potrebno, da kratka stran ima pripravljeno dejansko obveznico za prodajo ob datumu izročitve.
To je dovoljeno zato, ker vedno obstaja možnost predčasnega zapiranja pozicije. 

Če pa se dejansko odloči za izvedbo, pa mora predati obveznico. Le-to izbere iz množice obveznic, ki jih
CBOT določi za primerne. Eden izmed pogojev je, da imajo obveznice do datuma zapadlosti na datum izročitve
vsaj 15 let, drugače pa se zahteva, da so vsaj podobne navedeni hipotetični obveznici, ker ni nujno, da na
trgu obstaja identična obveznica. 

Ta aspekt je tudi ena izmed najbolj zanimivih stvari pri tej vrsti pogodbe, zato si bomo v nadaljevanju 
pogledali nekaj detajlov.

% mogoče en primer tukaj; tista tabela iz vira recimo

\subsubsection{Pretvorbeni faktorji}
Zaradi nenujosti obstoja primerne obveznice, pred prodajo pogodb z določenim datumom zapadlosti CBOT 
določi pretvorbene faktorje (angl. \textit{conversion factors}), ki naredijo te pogodbe pravične. 
Odvisni so od dejanske obveznice, ki jo dolga stran da, in od izročitvenega datuma. Če torej prodajalec
da dejansko bolj donosno obveznico, kot je v pogodbi, bo CBOT tako zagotovil, da vseeno dobi pravično
ceno (in obratno, če da manj donosno obveznico).

\begin{primer}
    Recimo, da imamo pogodbo za obveznico z vrednostjo \textdollar$100.000$ in kuponsko mero $6\$$. 
    Če nam prodajalec da obveznico z enako vrednostjo a višjo kuponsko mero; recimo $7\%$, to ni 
    pošteno, saj je taka obveznica bolj vredna. Če pa je kuponska mera $5\%$, pa spet ni pravično, 
    zato s pretvorbenim faktorjem izenačimo cene.
\end{primer}

Dejanska cena, ki jo kupec plača (angl. \textit{invoice price}), je tako odvisna od mnogih stvari. 
Seveda je pomembna izročitvena cena, določena v sami pogodbi in ">velikost pogodbe"< (torej število
obveznic, ki se prodajo s pogodbo). Zgoraj smo ugotovili, da je pomemben tudi pretvorbeni faktor. 
V osnovi pa govorimo o instrumentih na obrestne mere, zato ne smemo pozabiti na natečene obresti, 
ki jih prinaša naše osnovno premoženje. Če povzamemo:

\begin{align*}
    cena 
    &= št.\:obveznic \times K \times pretvorbeni\:faktor + natečene\:obresti \\
    &= št.\:obveznic \times pretvorjena\:cena + natečene\:obresti
\end{align*}

\subsubsection{Obveznica, najcenejša za poravnavo}
Zgoraj smo ugotovili, da je možno pri predaji izbirati med različnimi obveznicami, in seveda niso
vse enakovredne. Želja dolge strani je poiskati najcenejšo (angl. \textit{cheapest-to-deliver
issue}) in zato je potrebna analiza vseh možnosti. To izvedemo z naslednjim razmislekom:

Zamislimo si, da istočasno prodamo terminsko pogodbo in z izposojenim denarjem kupimo eno 
izmed obveznic, ki so navedene kot primerne za to pogodbo. Na datum izročive potem predamo 
obveznico in z dobljenim denarjem poplačamo dolg - to strategijo imenujemo \textit{cash 
and carry trade}. S tako situacijo potem poznamo dovolj podatkov, da izračunamo stopnjo 
povračila za tako investicijo, v angleščini to imenujemo \textbf{implied repo rate}. To je 
torej implicirana obrestna mera oz. impliciran donos, ki ga ima ta strategija in s tem tudi 
naša potencialna obveznica. Tak izračun lahko naredimo za vse primerne obveznice in na 
koncu izberemo tisto, katere donos je najvišji. To obveznico potem imenujemo obveznica 
najcenejša za poravnavo (angl. \textit{cheapest-to-deliver issue}). Ta ima ključno vlogo
pri vrednotenju terminskih pogodb o obrestni meri (več o tem si lahko preberete v kasnejši temi).

Za izračun impliciranega donosa moramo o obveznici poznati naslednje podake:

\begin{itemize}
    \item Ceno po kateri lahko kupimo obveznico in natečene obresti, ki ji pripradajo,
    \item Pretvorjeno ceno in natečene obresti, ki jih bomo dobili ob izročitvi, 
    \item Kupone, ki jih bomo dobili do izročitvenega datuma,
    \item Dobiček od reinvesticije kuponov od njihovega izplačila do datuma izročive.
\end{itemize}

Čeprav slednje točke ne moremo točno poznati, imajo njene podrobnosti majhen učinek na 
implicirano stopnjo povračila in zato jih lahko zanemarimo. S tem dobimo naslednjo formulo:
%
$$ impliciran\:donos = \frac{donos}{strošek\:investicije} \times
\frac{360}{dnevi\:do\:izročitve}, $$
%
kjer je donos razlika med prejetim izkupičkom ob izročitvenem datumu in stroškom investicije.
Izkupiček je torej izročitvena cena terminske pogodbe (ne pozabimo upoštevati pretvorbenega 
faktorja in natečenih obresti), kuponi, ki jih prejmemo in obresti od reinvesticije kuponov:
%
$$ izkupiček = pretvorjena\:cena + natečene\:obresti + kuponi + obresti, $$
%
kjer velja, da obresti ne poznamo točno, pač pa to ocenimo glede na tržne razmere. Pripomnimo
še, da kupone reinvestiramo za toliko dni, kot mine med njihovim izplačilom in izročitvenim
datumom in da to v formuli merimo kot delež leta (t.j. $360$ dni). 

Strošek investicije je pa torej cena obveznice, ki jo kupimo in natečene obresti, ki jih plačamo.

Zadnji ulomek pa je enostavno anualizacija te repo stopnje, saj to nekako standardizira izračun
in omogoča lažjo primerjavo. 

\begin{primer}
    Recimo, da imamo terminsko pogodbo z izročitveno ceno $K = 96$ in s $82$ dnevi do izročitvenega
    datuma. poznamo tudi $42$-dnevno terminsko obrestno mero, po kateri lahko reinvestiramo kupon.
    Ta je je $3,8\%$.

    Hkrati pa imamo na voljo primerno obveznico z naslednjimi podatki:

    \begin{itemize}
        \item cena $P = 107$
        \item plačane natečene obresti $3,8904$
        \item kuponska obrestna mera $c = 10\%$
        \item čas do izplačila kupona $ = 40$ dni
        \item kupon $C = \textdollar5$
        \item prejete natečene obresti ob izročitvi $ = 1,1507$
        \item pretvorbeni faktor $ = 1,1111$
    \end{itemize}

    Najbolj enostaven je izračun pretvorjene cene:
    %
    $$ pretvorjena\:cena = P \times pretvorbeni\:faktor = 96 \times 1.1111 = 106,6656 $$
    %
    Za izračun impliciranega donosa potrebujemo še obresti od reinvesticije kupona, ki jih
    dobimo po naslednji formuli:
    
    \begin{align*}
        obresti
        &= kupon \times obrestna\:mera \times \frac{št.\:dni\:vezave}{360} \\
        &= \textdollar5 \times 0,038 \times \frac{82 - 40}{360} = 0,0222 
    \end{align*}

    Kjer je število dni vezave čas med izplačilom kupona in izročitvenim datumom. 

    To nam da dovolj podatkov za izračun izkupička kot vsote pretvorjene cene, prejetih natečenih
    obresti, kuponom, ki ga prejmemo, in obrestmi oz. $106,6566 + 1,1507 + 5 + 0,0222 = 112,8385$

    Strošek investicije je cena obveznice in plačane natečene obresti oz. $107 + 3,8904 = 110,8904$.

    Zdaj končno lahko izračunamo impliciran donos:
    %
    $$ impliciran\:donos = \frac{112,8385 - 110,8904}{110,8904} \times \frac{360}{82} 
    = 7,71\% $$
    
\end{primer}

\subsubsection{Izbire pri izročitvi}
Prodajalec ima ob izročitvi pravico in dolžnost sprejeti nekaj pomembnih odločitev. CBOT mu 
zagotavlja, da se lahko odloča vsaj glede naslednjih stvari:

\begin{itemize}
    \item Odločiti se mora, katero izmed primernih obveznic bo izročil. Tej odločitvi v angleščini 
            rečemo \textit{quality/swap option}.
    \item Izbere lahko, kateri dan v izročitvenem mesecu bo izkoristil za dejansko izročitev
            obveznice - \textit{timing option}.
    \item Lahko si izbere, da obveznico preda pred ali po zaprtju borze; tako lahko izbere, ali bo 
            znana končna cena pogodbe ob predaji. Odločitev je pogosto znana pod imenom
            \textit{wild card option}. 
\end{itemize}

Zaradi teh odločitev dolga pozicija nikoli tekom pogodbe ne more vedeti točno, katero obveznico bo
dobila, niti ne pozna točnega datuma predaje.

\subsubsection{Proces izročitve}
Ko se pogodba izteče (oz. si kratka stran izbere datum), prodajalca še vedno čaka 3-dnevni proces 
izročitve. 

Prvi dan (angl. \textit{position day}) mora do osme ure zvečer po chicaškem času sporočiti na 
CBOT, da ima namen izvršiti pogodbo. 

Drugi dan (angl. \textit{notice day}) si izbere, katero obveznico bo predal. To odločitev mora 
sprejeti do štirinajste ure. CBOT potem pregleda vse potencialne kupce, izbere tistega, ki 
ima odprto dolgo pozicijo najdlje in mu do šestnajste ure sporoči, da se bo pogodba izvršila. 

Tretji dan (angl. \textit{delivery day}) se do desete ure izvede dejanska izročitev. Prodajalec 
mora imeti na računu pripravljeno obveznico, kupec pa denar. 


\subsection{Terminske pogodbe na zakladno menico \\ (angl. \textit{treasury note futures})}
CBOT ponuja pogodbe dolžine $2$, $5$ ali $10$ let, ki so vse izpeljanke standardnih pogodb na 
državne obveznice in se še vedno menjajo primarno pri njih. Za vsako ročnost pogodbe je 
definirana malce drugačno hipotetično osnovno premoženje:

\begin{itemize}
    \item \textit{za $10$-letno pogodbo:} je osnovno premoženje $10$-letna menica s $6$-procentno
            kuponsko obrestno mero in vrednostjo \textdollar$100.000$. Kratka stran lahko dostavi 
            obveznico, ki ima ročnost med $6.5$ in $10$ leti od prvega dne v izročitvenem mesecu.
    \item \textit{za $5$-letno pogodbo:} je osnovno premoženje menica s $6$-procentno kuponsko 
            obrestno mero in vrednostjo \textdollar$100.000$, ki ima skupno ročnost vsaj $5$ let
            in $3$ mesece, od katere je preostalo vsaj $4$ leta in $2$ meseca.
    \item \textit{za $2$-letno pogodbo:} je osnovno premoženje menica s $6$-procentno kuponsko 
    obrestno mero in vrednostjo \textdollar$200.000$, ki ima skupno ročnost manj kot $5$ let
    in $3$ mesece, od katere je preostalo med $1$ letom in $9$ mesecev in $2$ letoma.
\end{itemize} 

\subsubsection{Opomba glede menice in obveznice}
V osnovi je edina razlika med menico in obveznico ročnost. Obveznice imajo ročnost več kot $10$
let, menice pa manj kot $10$ let. Pripomnimo še, da je v Sloveniji delitev drugačna. Menice imajo
ročnost manj od leta, vse ostalo so obveznice.

\subsection{Za kaj se še uporabljajo terminske pogodbe o obrestni meri?}
Zgoraj opisane pogodbe se prodajajo predvsem v ZDA. V Evropi so bolj popularne terminske pogodbe, ki 
za osnovno premoženje instrumente, ki temeljijo na Eurodolarju in obrestnih merah EURIBOR in LIBOR.

\section{Viri}
\begin{itemize}
    \item Frank. J. Fabozzi: \textit{Fixed Income Analysis}, John Wiley \& Sons $2$. izdaja, 2007
    \item Investopedia: \textit{Interest Rate Future} (leto ogleda: $2020$), \\
            dostopno na: \url{https://www.investopedia.com/terms/i/interestratefuture.asp}
            \item AccountingTools: \textit{Interest Rate Futures} (leto ogleda: $2020$), \\
            dostopno na: \url{https://www.accountingtools.com/articles/2017/5/15/interest-rate-futures}
\end{itemize}

\end{document}