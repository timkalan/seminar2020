\documentclass[14pt]{beamer}
\usepackage[T1]{fontenc}
\usepackage[utf8]{inputenc}
\usepackage[slovene]{babel}
\usepackage{pgfpages} % privat zapiski

%\setbeameroption{hide notes}                        % samo prosojnice
%\setbeameroption{show only notes}                   % samo zapiski
\setbeameroption{show notes on second screen=right}  % oboje

% Naj se pojavljajo zaporedoma:
% \beamerdefaultoverlayspecification{<+->}

% Minimalistični still
\mode<presentation>
% \usetheme{Goettingen}
% \usecolortheme{rose}

\usepackage{palatino}
\usefonttheme{serif}

\setbeamertemplate{navigation symbols}{} % izklop navigacije
\setbeamertemplate{footline}[frame number]{} % oštevilčenje
\setbeamertemplate{note page}{\pagecolor{yellow!5}\insertnote}

\author{Tim Kalan \\ 
    Fakulteta za matematiko in fiziko}
\title{
    Terminske pogodbe o obrestni meri \\ 
    \large (angl. \textit{Interest rate futures})}
\date{\today} 

\begin{document}


\begin{frame}
    \titlepage
\end{frame}


\begin{frame}
    \frametitle{Uvod}

    \begin{itemize}
        \item Zakaj trgujemo s finančnimi instrumenti o obrestni meri?
    \end{itemize}

    \note[item]{cena, hitrost, likvidnost}

\end{frame}


% cel del o trgovanju mogoče lahko en slajd


\begin{frame}
    \frametitle{Trgovanje s terminskimi pogodbami - Terminske pogodbe}
    
    \begin{itemize}
        \item Natančen dogovor med kupcem in prodajalcem
        \item Določi se vrsta blaga, datum izročitve, izročitvena cena
        \item Osnovno premoženje praktično karkoli 
    \end{itemize}

    \note[item]{Organiziran trg -> standardizirane}
    \note[item]{Dolga, kratka stran, angleški izrazi?}

\end{frame}


\begin{frame}
    \frametitle{Trgovanje s terminskimi pogodbami - Zapiranje pozicije}
    
    \begin{itemize}
        \item Pred zapadlostjo: obratna pozicija
        \item Zapadlost: izvedemo obveznosti
    \end{itemize}

    \note[item]{Marec, junij, september, december}
    \note[item]{Pred: Kupimo/prodamo isto število obratnih pogodb}
    \note[item]{Na: Izročimo blago/plačamo, kar smo dolžni}

\end{frame}


\begin{frame}
    \frametitle{Trgovanje s terminskimi pogodbami - Vloga klirinške hiše}
    
    \begin{itemize}
        \item ">Vrine"< kot kupec oz. prodajalec
        \item Zmanjša tveganje in poenostavi zapiranje pozicije
    \end{itemize}

    \note[item]{Objasni kaj sploh je klirinška hiša (middle man)}
    \note[item]{Po dogovoru ni več stika med nasprotima stranema pogodbe}

\end{frame}


\begin{frame}
    \frametitle{Trgovanje s terminskimi pogodbami - Zahteve po kritju}
    
    \begin{itemize}
        \item Vzdrževalni račun, začetno kritje, vzdrževalno kritje
        \item Zmanjša tveganje 
    \end{itemize}

    \note[item]{Obrazložiš pač marking-to-market, nihanje cen, ...}
    \note[item]{Začetno kritje lahko karkoli z obrestmi, kasdneje denar in denarni ustrezniki}

\end{frame}


\begin{frame}
    \frametitle{Terminski posli}
    
    \begin{itemize}
        \item Nestrukturirane terminske pogodbe
        \item Neorganiziran trg (\textit{OTC})
        \item Neobstoječ sekundarni trg
    \end{itemize}

    \note[item]{marking to market}
    \note[item]{bilateral counterparty risk - ni klirinške}

\end{frame}


\begin{frame}
    \frametitle{Terminske pogodbe o obrestni meri}
    
    \begin{itemize}
        \item Osnovno premoženje izplačuje obresti
        \item Varnost pred spremembami obrestnih mer (\textit{hedging}), s 
                špekulacijo tudi zaslužek
        \item CBOT; državne obveznice, zakladne menice
    \end{itemize}

    \note[item]{kratko- in dolgotrajne pogodbe; eno leto}

\end{frame}


\begin{frame}
    \frametitle{Terminske pogodbe na državne obveznice}
    
    \begin{itemize}
        \item Osnovno premoženje hipotetična dvajset-letna kuponska 
                obveznica z vrednostjo $100.000$
        \item Kotacija: $100\%$ vrednosti, razlike v dvaintridesetinah
        \item Minimalno odstopanje cene je $1 / 32 \%$
    \end{itemize}

    \note[item]{Nedokončano}
    \note[item]{kaj je minimum price flux?}

\end{frame}


\begin{frame}
    \frametitle{Hipotetične obveznice}
    
    \begin{itemize}
        \item Kratka stran pri vstopu nima obveznice
        \item CBOT določi primerne kandidate
        \item To omogoča strategiranje in ekonomsko analizo
    \end{itemize}

    \note[item]{NI treba da jo ima, ker lahko predčasno zapre}
    \note[item]{Pač zato je zanimivo sploh to}

\end{frame}


\begin{frame}
    \frametitle{Pretvorbeni faktorji}
    
    \begin{itemize}
        \item Naredijo pogodbe pravične - če predamo bolj donosno
                obveznico, vseeno dobimo pravično ceno
        \item Pač dej kaj so te natečene zdej
        \item Dejanska cena tako odvisna od mnogih stvari:
    \end{itemize}
    %
    $$ cena = št.\:obveznic \times K \times pretvorbeni\:faktor + natečene\:obresti $$
    %

    \note[item]{Primer tukaj ane?}
    \note[item]{Spet CBOT določi}

\end{frame}


\begin{frame}
    \frametitle{Obveznica, najcenejša za dostavo}
    
    \begin{itemize}
        \item 
    \end{itemize}

    \note[item]{}

\end{frame}


\begin{frame}
    \frametitle{Implicirana repo stopnja}
    
    \begin{itemize}
        \item 
    \end{itemize}

    \note[item]{}

\end{frame}


\begin{frame}
    \frametitle{Viri}
    
    \begin{itemize}
        \item Vir 
        \item Investipedia
    \end{itemize}

\end{frame}

\end{document}