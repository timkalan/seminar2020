\documentclass[14pt]{beamer}
\usepackage[T1]{fontenc}
\usepackage[utf8]{inputenc}
\usepackage[slovene]{babel}
\usepackage{pgfpages} % privat zapiski
\usepackage{amsmath} % pravilen izpis v "math mode"
\usepackage{hyperref}
\hypersetup{hidelinks}

\setbeameroption{hide notes}                        % samo prosojnice
%\setbeameroption{show only notes}                   % samo zapiski
%\setbeameroption{show notes on second screen=right}  % oboje

\usepackage{palatino}
\usefonttheme{serif}

\setbeamertemplate{navigation symbols}{} % izklop navigacije
\setbeamertemplate{footline}[frame number]{} % oštevilčenje
\setbeamertemplate{note page}{\pagecolor{yellow!5}\insertnote}

\author{Tim Kalan \\ 
    Fakulteta za matematiko in fiziko}
\title{
    Terminske pogodbe o obrestni meri \\ 
    \large (angl. \textit{Interest rate futures})}
\date{14.\ april\ 2020} 

\begin{document}


\begin{frame}
    \titlepage
\end{frame}


\begin{frame}
    \frametitle{Uvod}

    \begin{itemize}
        %\item Obveznice, menice, Eurodollar, EURIBOR, LIBOR
        \item Zakaj trgujemo s finančnimi instrumenti o obrestni meri?
        
    \end{itemize}

    \note[item]{cena, hitrost, likvidnost}

\end{frame}


% cel del o trgovanju mogoče lahko en slajd


\begin{frame}
    \frametitle{Trgovanje s terminskimi pogodbami - Terminske pogodbe}
    
    \begin{itemize}
        \item Natančen dogovor med kupcem in prodajalcem (standardizirano)
        \item Določi se vrsta blaga, datum izročitve, izročitvena cena
        \item Osnovno premoženje praktično karkoli 
        \item Finančne terminske pogodbe
    \end{itemize}

    \note[item]{Organiziran trg -> standardizirane}
    \note[item]{Dolga, kratka stran, angleški izrazi?}

\end{frame}


\begin{frame}
    \frametitle{Trgovanje s terminskimi pogodbami - Zapiranje pozicije}
    
    \begin{itemize}
        \item Pred zapadlostjo: obratna pozicija
        \item Zapadlost: izvedemo obveznosti
    \end{itemize}

    \note[item]{Marec, junij, september, december}
    \note[item]{Pred: Kupimo/prodamo isto število obratnih pogodb}
    \note[item]{Na: Izročimo blago/plačamo, kar smo dolžni}

\end{frame}


\begin{frame}
    \frametitle{Trgovanje s terminskimi pogodbami - Vloga klirinške hiše}
    
    \begin{itemize}
        \item ">Vrine"< kot kupec oz. prodajalec
        \item Zmanjša tveganje in poenostavi zapiranje pozicije
        \item T. P. je med investitorjem in klirinško hišo
    \end{itemize}

    \note[item]{Objasni kaj sploh je klirinška hiša (middle man)}
    \note[item]{Po dogovoru ni več stika med nasprotima stranema pogodbe}

\end{frame}


\begin{frame}
    \frametitle{Trgovanje s terminskimi pogodbami - Zahteve po kritju}
    
    \begin{itemize}
        \item Vzdrževalni račun
        \item Začetno, vzdrževalno, variacijsko kritje
        \item Zmanjša tveganje 
    \end{itemize}

    \note[item]{Obrazložiš pač marking-to-market, nihanje cen, ...}
    \note[item]{Začetno kritje lahko karkoli z obrestmi, kasdneje denar in denarni ustrezniki}

\end{frame}


\begin{frame}
    \frametitle{Terminski posli}
    
    \begin{itemize}
        \item Nestandardizirane terminske pogodbe
        \item Neorganiziran trg (\textit{OTC})
        \item Neobstoječ sekundarni trg
        \item Tveganje, vmesni denarni tokovi
    \end{itemize}

    \note[item]{marking to market}
    \note[item]{bilateral counterparty risk - ni klirinške}

\end{frame}


\begin{frame}
    \frametitle{Terminske pogodbe o obrestni meri}
    
    \begin{itemize}
        \item Osnovno premoženje izplačuje obresti
        \item Varnost pred spremembami obrestnih mer (\textit{hedging}), s 
                špekulacijo tudi zaslužek
        \item CBOT; državne obveznice, zakladne menice
    \end{itemize}

    \note[item]{kratko- in dolgotrajne pogodbe; eno leto}

\end{frame}


\begin{frame}
    \frametitle{Primer: \textit{Kako se zavarujemo?}}
    
    \begin{itemize}
        \item Izposojen denar po variabilni obrestni meri
        \item Prodamo terminsko pogodbo na obveznico
        \item Dvig mere: večje obresti, a bolj vredna tudi pogodba
        \item Rezultat: Začetna mera velja za celotno obdobje
    \end{itemize}

    \note[item]{}

\end{frame}


\begin{frame}
    \frametitle{Terminske pogodbe na državne obveznice}
    
    \begin{itemize}
        \item Osnovno premoženje hipotetična dvajset-letna kuponska 
                obveznica z vrednostjo \textdollar$100.000$
        \item Kotacija: $100\%$ vrednosti, razlike v dvaintridesetinah
        \item Minimalno odstopanje cene je $1 / 32 \%$
    \end{itemize}

    \note[item]{Nedokončano}
    \note[item]{kaj je minimum price flux?}

\end{frame}


\begin{frame}
    \frametitle{Primer: \textit{Kotacija}}
    
    \begin{itemize}
        \item $101-25$ za \textdollar$100.000$ obveznico
        \item $ (101-25) = 
            \textdollar100.000 + \textdollar1.000 + (\textdollar1.000 \times \frac{25}{32}) $
    \end{itemize}

    \note[item]{}

\end{frame}


\begin{frame}
    \frametitle{Hipotetične obveznice}
    
    \begin{itemize}
        \item Kratka stran pri vstopu nima obveznice
        \item CBOT določi primerne kandidate
        \item To omogoča strategiranje in ekonomsko analizo
    \end{itemize}

    \note[item]{NI treba da jo ima, ker lahko predčasno zapre}
    \note[item]{Pač zato je zanimivo sploh to}

\end{frame}


\begin{frame}
    \frametitle{Pretvorbeni faktorji}
    
    \begin{itemize}
        \item Naredijo pogodbe pravične - če predamo bolj donosno
                obveznico, vseeno dobimo pravično ceno
        \item Ne pozabimo na natečene obresti 
        \item Dejanska cena tako odvisna od mnogih stvari:
    \end{itemize}
    %
    \textit{$ \textbf{cena} = \text{št. obveznic} \times \text{K} \times \text{pretvorbeni\:faktor} 
        + \text{natečene obresti} $}
    %

    \note[item]{Primer tukaj ane?}
    \note[item]{Spet CBOT določi}

\end{frame}


\begin{frame}
    \frametitle{Obveznica, najcenejša za poravnavo}
    
    \begin{itemize}
        \item Kratka stran želi dati najcenejšo - analiza možnosti
        \item Prodamo terminsko pogodbo z izposojenim denarjem, 
                kupimo eno od primernih obveznic
        \item Izračunamo implicirane repo stopnje za vse primerne obveznice
    \end{itemize}

    \note[item]{cheapest-to-deliver issue}
    \note[item]{cash and carry trade}

\end{frame}


\begin{frame}
    \frametitle{Implicirana repo stopnja (\textit{IRS})}
    
    \begin{itemize}
        \item \textit{$ \textbf{IRS} = \frac{\text{donos}}{\text{strošek investicije}} 
                \times \frac{360}{\text{dnevi do izročitve}} $}
        \item \textit{$ \textbf{donos} = \text{izkupiček} - \text{strošek investicije} $}
        \item \textit{$ \textbf{izkupiček} = \text{pretvorjena cena} + \text{prejete natečene obresti} 
                + \text{kuponi} + \text{obresti} $}
        \item \textit{$ \textbf{obresti} =  \text{kupon} \times \text{obrestna mera} \times 
                \frac{\text{št. dni vezave}}{360} $}
        \item \textit{$ \textbf{strošek investicije} = \text{cena obveznice} + 
                \text{plačane natečene obresti}$}
    \end{itemize}

    \note[item]{katere obresti, kuponi, natečene obresti?}
    \note[item]{Pojasni da je to neka rekurzivna zadevščina}

\end{frame}


\begin{frame}
    \frametitle{Primer: Izračun \textit{IRS} 1}
    
    \begin{itemize}
        \item Terminska pogodba: 
        
                \begin{itemize}
                    \item Izročitvena cena $K = 96$
                    \item Dnevi do izročitvenega datuma $= 82$
                \end{itemize}

        \item Tržne razmere:
        
                \begin{itemize}
                    \item Terminska obrestna mera, po kateri lahko reinvestiramo
                            kupon $R = 3,8\%$
                \end{itemize}

        \item Obveznica:
        
                \begin{itemize}
                    \item cena $P = 107$
                    \item plačane natečene obresti $3,8904$
                    \item kuponska obrestna mera $c = 10\%$
                    \item čas do izplačila kupona $ = 40$ dni
                    \item prejete natečene obresti ob izročitvi $ = 1,1507$
                    \item pretvorbeni faktor $ = 1,1111$
                \end{itemize}

    \end{itemize}

    \note[item]{}

\end{frame}


\begin{frame}
    \frametitle{Primer: Izračun \textit{IRS} 2}
    
    \begin{align*}
        \textit{pretvorjena cena} 
        &= \textit{P} \times \textit{pretvorbeni faktor} \\
        &= 96 \times 1.1111 \\
        &= 106,6656
    \end{align*}
    
    \begin{align*}
        \textit{obresti}
        &= \textit{kupon} \times \textit{obrestna mera} \times 
            \frac{\textit{št. dni vezave}}{360} \\
        &= \textdollar5 \times 0,038 \times \frac{82 - 40}{360} \\
        &= 0,0222 
    \end{align*}

\end{frame}


\begin{frame}
    \frametitle{Primer: Izračun \textit{IRS} 3}
    
    \begin{align*}
        \textit{izkupiček} 
        &= \textit{pretvorjena cena} + \textit{prejete natečene obresti} \\
        &+ \textit{kupon} + \textit{obresti} \\
        &= 106,6566 + 1,1507 + 5 + 0,0222 \\
        &= 112,8385 
    \end{align*}
    
    \begin{align*}
        \textit{investicija}
        &= \textit{cena obveznice} + \textit{plačane natečene obresti} \\
        &= 107 + 3,8904 \\
        &=110,8904
    \end{align*}

    \note[item]{}

\end{frame}


\begin{frame}
    \frametitle{Primer: Izračun \textit{IRS} 4}

    \begin{align*}
        \textit{IRS} 
        &= \frac{\textit{donos}}{\textit{strošek investicije}} \times 
            \frac{360}{\textit{dnevi do izročitve}} \\
        &= \frac{112,8385 - 110,8904}{110,8904} \times \frac{360}{82} \\
        &= 7,71\% 
    \end{align*}

    \note[item]{}

\end{frame}


\begin{frame}
    \frametitle{Izbire pri izročitvi}
    
    \begin{itemize}
        \item Katera obveznica?
        \item Kater dan?
        \item Pred ali po zaprtju borze?
    \end{itemize}

    \note[item]{quality/swap, timing, wild card}

\end{frame}


\begin{frame}
    \frametitle{Proces izročitve}
    
    \begin{itemize}
        \item Prvi dan: namen
        \item Drugi dan: izbira obveznice, kupca
        \item Tretji dan: izročitev
    \end{itemize}

    \note[item]{position, notice, delivery}

\end{frame}


\begin{frame}
    \frametitle{Katere so ostale terminske pogodbe o obrestni meri?}
    
    \begin{itemize}
        \item Zakladna menica ($2$, $5$, $10$ letna )
        \item Vladne agencije (Fannie Mae, Freddie Mac)
        \item EURIBOR in LIBOR
        \item Eurodollar
    \end{itemize}

    \note[item]{position, notice, delivery}

\end{frame}


\begin{frame}
    \frametitle{Viri}
    
    \begin{itemize}
        \item Frank. J. Fabozzi: \textit{Fixed Income Analysis}, John Wiley \& Sons $2$. izdaja, 2007
        \item Investopedia: \textit{Interest Rate Future} (leto ogleda: $2020$), \\
                dostopno na: \url{https://www.investopedia.com/terms/i/interestratefuture.asp}
        \item AccountingTools: \textit{Interest Rate Futures} (leto ogleda: $2020$), \\
                dostopno na: \url{https://www.accountingtools.com/articles/2017/5/15/interest-rate-futures}
    \end{itemize}

\end{frame}

\end{document}